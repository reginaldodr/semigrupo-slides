\section{Espaços de Banach}

\subsection*{Espaços Normados}

\begin{frame}{Espaços Normados}
\begin{defin}
    Seja $X$ um $\mathbb K$-espaço vetorial. Uma aplicação
    \[
    \displaystyle \| \cdot \|: X\longrightarrow \R
    \]
    é dita uma {\color{blue} norma} se, para quaisquer $x,y\in X$ e $\lambda \in \mathbb K$, as seguintes condições se verificarem:
    \begin{enumerate}[(a)]
    \item $\|x\|\geq 0$;
    \item Se $\|x\|=0$, então $x=0$;
    \item $\|\lambda x\|=|\lambda| \|x\|$;
    \item $\|x+y\|\leq \|x\|+\|y\|$.
    \end{enumerate}
    Nesse caso, o par $(X,\|\cdot\|)$ é dito um \dt{espaço normado} \index{Espaço!normado}.
\end{defin}
\end{frame}
%%%%%%%%%%%%%%%%%%%%%%%%%%%%%%%%%%%%%
% \begin{frame}{Exemplos}
% \begin{enumerate}[a]
% \item $\mathbb{K}^n$ é um espaço normado com a norma:
% \[\displaystyle \|\cdot \|_0 : (x_1,\cdots , x_n) \in \mathbb K^n \longmapsto \bigg( \sum_{j=1}^{n} |x_j|^2\bigg)^{1/2} \in \R \]

% \item  O espaço das matrizes $m\times n$, denotado por $\mathcal M _{m\times n} (\mathbb K )$, é um espaço normado, considerando a norma
% \[
% \displaystyle \|A\|=\bigg( \sum_{j,k=1}^{n} |a_{jk}|^2\bigg)^{1/2},
% \]
% definida para cada $A=[a_{jk}]_{m\times n}\in \mathcal M _{m\times n} (\mathbb K )$. Na verdade, $\mathcal M _{m\times n} (\mathbb K )$ é um espaço de Banach se considerarmos qualquer outra norma ({\color{red} Exercício!}).    

% \item Denotemos por $\mathcal B (A)$ o conjunto de todas as funções limitadas $f: A\longrightarrow \mathbb K$.
% \[
% \displaystyle f\in \mathcal B (A) \longmapsto \| f\| = \sup_{x\in A} |f(x)| \in \R
% \]
% \end{enumerate}
% \end{frame}
%%%%%%%%%%%%%%%%%%%%%%%%%%%%%%%%%%%%%%%
\begin{frame}{Sequências em Espaços Normados}
\begin{defin}
 Sejam $X$ um espaço normado. Dizemos que uma sequência  $x=(x_n)_{n=1}^{\infty}$  em $X$
 %isto é, $x:\mathbb N \longrightarrow X$ é uma função que associa cada $n\in \mathbb N$ a um único $x_n \in X$. 
%Dizemos que $x=(x_n)_{n=1}^{\infty}$ 
\begin{enumerate}[(a)]
    \item {\color{blue} converge} para $\raa\in X$ se, para cada $\varepsilon >0$, existir $n_0 \in \mathbb N$ de modo que
    \begin{equation*}
    n\in \mathbb N \text{ e } n\geq n_0 \Longrightarrow \|x_n-\raa\|< \varepsilon .
\end{equation*}
\item é {\color{blue} de Cauchy} se, para cada $\varepsilon >0$, existir $n_1 \in \mathbb N$ de modo que
\begin{equation*}
    m,n\in \mathbb N \text{ e } m,n\geq n_1 \Longrightarrow \|x_m-x_n\|< \varepsilon .
\end{equation*}
\end{enumerate}
\end{defin}
\end{frame}
%%%%%%%%%%%%%%%%%%%%%%%%%%%%%%%%%%%%%%%
\begin{frame}
\begin{exer}
    \begin{enumerate}[(a)]
        \item Em um espaço normado $X$, mostre que toda sequência convergente é uma sequência de Cauchy;
        \item Exiba um espaço normado $Y$ no qual exista uma  sequência de Cauchy que não converge em $Y$ ({\color{red} veja o Exemplo A.14 das Notas do Minicurso}).
    \end{enumerate}
\end{exer}
\end{frame}
%%%%%%%%%%%%%%%%%%%%%%%%%%%%%%%%%%%%%%%
\begin{frame}{Espaços de Banach}
\begin{defin}
    Um espaço normado $X$ é dito um {\color{blue} espaço de Banach} se toda sequência de Cauchy em $X$ convergir para um elemento de $X$.
\end{defin}
\end{frame}
%%%%%%%%%%%%%%%%%%%%%%%%%%%%%%%%%%%%%%%
\begin{frame}{Exemplos:}
    \begin{enumerate}[a]
        \item Para cada inteiro $n\geq 1$, $\mathbb K^n$ é um espaço de Banach, considerando a norma
        \[
        \displaystyle \|x\|_{0}=\bigg( \sum_{j=1}^{n} |x_j|^2\bigg)^{1/2},
        \]
        definida para cada $x=(x_1,\cdots ,x_n)\in \mathbb K^n$. Na verdade, $\mathbb K ^n$ é um espaço de Banach se considerarmos qualquer outra norma ({\color{red} Exercício!}).
    \end{enumerate}
\end{frame}
%%%%%%%%%%%%%%%%%%%%%%%%%%%%%%%%%%%%%%%
\begin{frame}
\begin{enumerate}[b]
\item Dados dois inteiros positivos $m,n\in \mathbb N$, o espaço das matrizes $m\times n$, denotado por $\mathcal M _{m\times n} (\mathbb K )$, é um espaço de Banach, considerando a norma
\[
\displaystyle \|A\|=\bigg( \sum_{j=1}^{m} \sum_{k=1}^{n}
|a_{jk}|^2\bigg)^{1/2},
\]
definida para cada $A=[a_{jk}]_{m\times n}\in \mathcal M _{m\times n} (\mathbb K )$. Na verdade, $\mathcal M _{m\times n} (\mathbb K )$ é um espaço de Banach se considerarmos qualquer outra norma ({\color{red} Exercício!}).        
\end{enumerate}    
\end{frame}
%%%%%%%%%%%%%%%%%%%%%%%%%%%%%%%%%%%%%%%
%\begin{frame}
%\begin{enumerate}[c]
%\item Dados dois inteiros positivos %$m,n\in \mathbb N$, o espaço das %{\color{red} transformações lineares} de $\mathbb R ^m$ em $\mathbb R^n$, denotado por $\mathcal L (\mathbb R^m; \mathbb R^n)$, é um espaço de Banach, considerando a norma
%\[
%\displaystyle \|T\|=\sup_{x\neq 0} %\frac{\|T(x)\|}{\|x\|},
%\]
%definida para cada $T\in \mathcal L (\mathbb R^m ; \mathbb R^n)$. Na verdade, $\mathcal M _{m\times n} (\mathbb K )$ é um espaço de Banach se considerarmos qualquer outra norma ({\color{red} Exercício!}).        
%\end{enumerate}    
%\end{frame}
%%%%%%%%%%%%%%%%%%%%%%%%%%%%%%%%%%%%%%%
\begin{frame}{Aplicações Contínuas}
\begin{defin}
    Sejam $X$ e $Y$ dois espaços normados. Dizemos que uma função $f:X\longrightarrow Y$ é contínua em $a\in X$ se, para cada $\varepsilon >0$, existir $\delta >0$ tal que
\[
x\in X \text{ e } \|x-a\|_X <\delta \Longrightarrow \|f(x) - f(a)\|_Y < \varepsilon .
\]
\end{defin}
\end{frame}
%%%%%%%%%%%%%%%%%%%%%%%%%%%%%%%%%%%%%%%
\begin{frame}{Aplicações Lineares Contínuas}
\begin{defin}
     Sejam $X$ e $Y$ dois espaços normados. Denotaremos por $\mathcal L (X,Y)$ o espaço vetorial de todas as {\color{blue} aplicações lineares e contínuas} de $X$ em $Y$, com as operações de adição e multiplicação por escalar definidas pontualmente. Cabe observar que:
    \begin{enumerate}[(a)]
    \item Quando $X=Y$, escreveremos {\color{red} $\mathcal L (X)$} em vez de $\mathcal L (X,X)$;
    \item Quando $Y=\mathbb K$, denotaremos $\mathcal L (X,\mathbb K)$ por {\color{red} $X'$}, que é conhecido como \dt{o dual topológico} \index{dual topológico} de X;
    \item O conjunto de todos os funcionais lineares $\varphi : X\longrightarrow \mathbb K$, contínuos ou não, será denotado por {\color{red} $X^{\ast}$}, que é conhecido como \dt{o dual algébrico} \index{dual algébrico} de X {\color{red} (veja o Apêndice A.3 das Notas do Minicurso).}
    \end{enumerate}
\end{defin}
\end{frame}
%%%%%%%%%%%%%%%%%%%%%%%%%%%%%%%%%%%%%%%
\begin{frame}{Aplicações Lineares Limitadas}
Dados dois espaços normados $X$ e $Y$, e uma aplicação linear $T:X\longrightarrow Y$, temos a seguinte equivalência:
{\color{red}
\[
\displaystyle T \text{ é contínua } \Longleftrightarrow \sup_{x\neq 0} \frac{\|T(x)\|_Y}{\|x\|_X} <+\infty.
\]
}
\begin{defin}
    Sejam $X$ e $Y$ dois espaços normados. Uma aplicação linear $T:X \longrightarrow Y$ é dita {\color{blue} limitada} 
se 
\begin{equation}\label{ltdo}
\sup_{x\neq 0} \frac{\|T(x)\|_Y}{\|x\|_X} 
<\infty.
\end{equation}
Em outras palavras, {\color{red} uma aplicação linear é limitada se, e somente se, é contínua.}
\end{defin}

\end{frame}
%%%%%%%%%%%%%%%%%%%%%%%%%%%%%%%%%%%%%%%
\begin{frame}{Exemplo:}
Dados dois espaços normados $X$ e $Y$, a aplicação
\[
\displaystyle T\in \mathcal L (X;Y) \longmapsto \sup_{x\neq 0} \frac{\|T(x)\|_Y}{\|x\|_X} \in \mathbb R
\]
é uma {\color{red} norma em $\mathcal L (X;Y)$}, denotada por $\|T\|_{\mathcal{L}(X;Y)}$. Em particular, 
\[\|T(x)\|_Y\leq \|T\|_{\mathcal{L}(X;Y)} \|x\|_X\]



É conhecido que
\begin{center}
\begin{minipage}{0.7\textwidth}
\begin{block}{}
{\color{blue}$\mathcal{L} (X;Y)$ {é um espaço de Banach} $\Longleftrightarrow Y$ {é um espaço de Banach}.}
\end{block}
\end{minipage}
\end{center}
\end{frame}
%%%%%%%%%%%%%%%%%%%%%%%%%%%%%%%%%%%%%%%
%\begin{frame}
%\begin{enumerate}[e]
%\item Denotemos por $\textbf{c}_{00}$ o $\mathbb K$-espaço vetorial de todas as sequências $(x_n)_{n=1}^{\infty}$ em $\mathbb K$ que possuem a seguinte propriedade: existe um inteiro positivo $n_0$ tal que $x_n=0$ para todo $n\geq n_0$. Considerando a norma
%\[
%\displaystyle \|x\| = \max_{j\in %\mathbb N} |x_j|,
%\]
%definida para cada $x=%(x_j)_{j=1}^{\infty} \in \textbf{c}_{00}$, temos um exemplo de um {\color{blue} espaço normado que não é um espaço de Banach} ({\color{red} Veja o Exemplo A.14 das Notas do Minicurso}).
%\end{enumerate}
%\end{frame}
%%%%%%%%%%%%%%%%%%%%%%%%%%%%%%%%%%%%%%%
% \subsection*{Operadores Lineares Ilimitados}

% \begin{frame}{Operadores Lineares ilimitados}
% Sejam $X$ e $Y$ dois espaços de Banach. Seja $A:D(A)\subset X\longrightarrow Y$ um operador linear,  onde $D(A)$ é um subespaço de $X$, chamado de {\color{blue} domínio} de $A$.
% \medskip
% Dizemos que $A$ é {\color{blue}limitado (ou contínuo)} se $D(A)=X$ e se existe  $C>0$ tal que 
% \begin{equation}\label{ltdo}
% \|Ax\|_{Y}\leq C\|x\|_X, \ \forall x\in D(A).
% \end{equation}
% Denotaremos por $\mathcal{L}(X,Y)$ o espaço de Banach dos {\color{blue} operadores lineares limitados} com norma dada por
% \[\|A\|_{_{\mathcal{L}(X,Y)}}=\sup\limits_{x\in X, x\neq 0}\frac{\|Ax\|_{_Y}}{\|x\|_{_X}},\]
% por $\mathcal{L}(X)=\mathcal{L}(X,X)$ e $X'=\mathcal{L}(X,\R)$.
% $A$ é dito ser {\color{blue}ilimitado} quando não satisfaz \eqref{ltdo}. Dizemos que $A$ é {\color{blue}densamente definito} se $\overline{D(A)}=X$.
% \end{frame}

%\subsection*{Integral Vetorial}
%
%\begin{frame}{Integral Vetorial}
%\begin{defin}
%Sejam $X$ um espaço de Banach e $u:[a,b]\longrightarrow X$ uma aplicação tal que, para cada $\varphi\in X'$,  a função real
%\[t\in [a,b] \longmapsto \langle\varphi, u(t) \rangle_{X'X}\in \R,\]
%seja integrável. Dizemos que $u$ é integrável se existe {\color{blue} um vetor $v\in X$} que satisfaz:
%\[\langle \varphi, v\rangle_{X',X}=\int_a^b \langle\varphi, u(t) \rangle_{X'X}\,dt,\ \forall \varphi\in X'.\]
%Em caso afirmativo, $v$ é único e escrevemos
%\[v=\int_a^b u(t)\,dt.\]
%
%\end{defin}
%\end{frame}
%
%\begin{frame}{ }
%\begin{prop}
%Se $u:[a,b]\longrightarrow X$ é {\color{blue}contínua}, então $u$ é integrável. Além disso, 
%\begin{enumerate}
%    \item $\displaystyle\left\|\int_a^b u(t)\,dt\right\|\leq \int_a^b \|u(t)\|\,dt$
%    \item Se $A\in \mathcal{L}(X,Y)$, então
%    \[ A\left(\int_a^b u(t)\,dt\right)=\int_a^b A(u(t))\,dt. \]
%\end{enumerate}
%
%\end{prop}
%\end{frame}
