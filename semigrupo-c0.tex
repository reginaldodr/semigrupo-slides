\section{Semigrupos de Classe $C^0$}


\subsection*{Definição de Semigrupo}
\begin{frame}{Semigrupos de Classe $C^0$}
\begin{defin}
    Seja $X$ um espaço de Banach. Dizemos que a aplicação $S:[0,+\infty)\longrightarrow \mathcal{L}(X)$ é um {\color{blue} semigrupo de operadores limitados em X} quando:
\begin{enumerate}
    \item $S(0)=\operatorname{Id}$;
    \item $S(t+s)=S(t)S(s)$, $\forall t,s\in [0,+\infty)$;

Dizemos que $S$ é {\color{blue} de classe $C^0$ ou fortemente contínuo} se

    \item $\lim\limits_{t\to0^+} \|(S(t)-\operatorname{Id})x\|_{X}=0$, $\forall x\in X$.

Dizemos que $S$ é {\color{blue} uniformemente contínuo} se 

\item $\lim\limits_{t\to 0^+}\|S(t)-\operatorname{Id}\|_{\mathcal{L}(X)}=0$.
\end{enumerate}
\end{defin}
\end{frame}

\begin{frame}{}
    \begin{exe}
        \begin{enumerate}
            \item A exponencial $e^{tA}$, quando $A\in \mathcal{L}(X)$, é um semigrupo.
            \item Seja $X=C_b(\R)$  o espaço das funções $f:\R\longrightarrow \R$ uniformemente contínuas e limitadas, com a norma do $\sup$, isto é,
            \[
            \displaystyle \|f\|:=\sup_{x\in \mathbb R} |f(x)|
            \]
            Então $S:[0,+\infty) \longrightarrow \mathcal L (X)$ dada por $[S(t)f](s)=f(t+s)$, onde $t\in [0,+\infty)$, $s\in \mathbb R$ e $f\in X$, define um semigrupo de classe $C^0$.
        \end{enumerate}
    \end{exe}

\end{frame}
%%%%%%%%%%%%%%%%%%%%%%%%%%%%%%%%%%%%%%%%%%%%%%%%%
\begin{frame}{Princípio da Limitação Uniforme}
\begin{teo}[Banach-Steinhaus]
Sejam $X$ e $Y$ dois espaços normado, com $X$ completo, e consideremos uma família
    \[
    \displaystyle \mathcal F =\{T_i :X\longrightarrow Y; i\in I\} \subset \mathcal L (X,Y),
    \]
    {\color{red} não necessariamente enumerável}, com a seguinte propriedade: para cada $x\in X$, temos
    \[
    \sup_{i\in I}\|T_ix\|_{Y} < \infty .
    \] 
    Então 
    \[
    \sup_{i\in I}\|T_i\|_{\mathcal{L}(X,Y)}<\infty.
    \]
    Em outras palavras, a limitação pontual da família $\mathcal F$ implica a sua limitação uniforme. 
    \end{teo}
\end{frame}
%%%%%%%%%%%%%%%%%%%%%%%%%%%%%%%%%%%%%%%%%%%%%%%%%
\begin{frame}{}
\begin{prop}
  Se $S$ é um semigrupo de classe $C^0$ em X, então existem $\mu\geq 0$ e $M\geq 1$ tais que 
    \begin{equation}\label{des.wT}
    \|S(t)\|_{\mathcal{L}(X)}\leq Me^{\mu t},\ \forall t\geq 0.  
    \end{equation}    
    Em particular, $\|S(t)\|_{\mathcal{L}(X)}$ é uma função limitada em todo intervalo $[0,T]$. 
\end{prop}
\begin{corol}
Todo semigrupo de classe $C^0$ é fortemente contínuo em $[0,+\infty)$, i.e., 
para todo $x\in X$, 
\begin{center}
$t\in [0,+\infty)\longmapsto S(\cdot)x\in  X$ é contínua.
\end{center}
\end{corol}
\end{frame}

\subsection*{Semigrupo das contrações}
% \begin{frame}{}
% \begin{teo}
% Seja $S$ um semigrupo de classe $C^0$ em $X$. Então,
% \[\lim\limits_{t\to +\infty}\frac{\log(\|S(t)\|)}{t}=\inf\limits_{t>0}\frac{\log(\|S(t)\|)}{t}=\omega_0\]
% e para cada ${\color{red}\omega}>\omega_0$, existe {\color{red}$M\geq 1$} tal que 
% \begin{equation*}
% {\color{red}\|S(t)\|\leq Me^{\omega t}, t\geq 0.}
% \end{equation*}
% \end{teo} 
% Quando $\omega_0<0$, então para $\omega=0$, temos que 
% \[\|S(t)\|\leq M,\ \forall t\geq 0. \]
% Neste caso, dizemos que $S$ é um {\color{blue} semigrupo uniformemente limitado}. Se, além disso, $M=1$, $S$ é dito {\color{blue} semigrupo das contrações}.

% \end{frame}

% \begin{frame}{ }
% \begin{lema}
%     Seja $p:[0,+\infty)\longrightarrow \R$ uma função {\color{blue}subaditiva}, isto é, $p(t+s)\leq p(t)+p(s)$. Se $p$ é limitada superiormente em todo intervalo limitado, então $p(t)/t$ tem um limite quanto $t\to +\infty$ e 
%     \[\lim\limits_{t\to +\infty}\frac{p(t)}{t}=\inf\limits_{t>0}\frac{p(t)}{t}.\]
% Prova: Ver \cite[Lema 1.2.5]{alvercio} 
% \end{lema}
% \end{frame}

\subsection*{Gerador Infinitesimal}
\begin{frame}{Gerador Infinitesimal }

\begin{defin}
Seja $S$ um semigrupo de classe $C^0$ em $X$. O {\color{blue}gerador infinitesimal} de $S$ é o operador $\rA:D(\rA)\subset X\longrightarrow X$ definido por 
\[D(\rA)=\left\{x\in X;\ \lim\limits_{h\to 0^+} \frac{S(h)x-x}{h} \text{ existe}\right\}\]
\[\rA x:=\lim\limits_{h\to 0^+} \frac{S(h)x-x}{h},\ \forall x\in D(\rA).\]
\end{defin}
\begin{prop}
    $D(\rA)$ é um subespaço vetorial de $X$ e $\rA$ é um operador linear.
\end{prop}

\end{frame}

% \begin{frame}{ }
% \begin{exampleblock}{Notações}
% \begin{enumerate}
%     \item Dado $S$ é um semigrupo de classe $C^0$ em $X$, vamos designar por $A_h$ o operador linear limitado 
%     \[A_hx=\frac{S(h)x-x}{h},\ \forall x\in X.\]
% \item Escrevemos $S(t)=e^{tA}$ para dizer que $A$ é o gerador infinitesimal de um semigrupo de classe $C^0$ em $X$.
% \item Escrevemos $A\in G({\color{blue}M},{\color{red}\omega})$ para exprimir que $A$ é o gerador infinitesimal de um semigrupo de classe $C^0$, que satisfaz a condição:
% \[\|e^{tA}\|\leq {\color{blue} M}e^{{\color{red}\omega} t}, \forall t\geq 0.\]
% \end{enumerate}
% \end{exampleblock}
% \end{frame}


\subsection*{Existência e Unicidade I}
\begin{frame}{ }
\begin{exampleblock}{Existência e Unicidade de um PVI}
Seja $S(t)$ um semigrupo de classe $C^0$ em $X$ e $\rA$ o seu gerador infinitesimal. Se {$\xo\in D(\rA)$}, então $x(t)=S(t)\xo$ define {\color{blue}uma única solução} do PVI
\[
\begin{cases}
    x'(t)=\rA x,\ t\in [0,+\infty)\\
    x(0)=x_0.
\end{cases}
\]
Além disso, 
\[x\in C^0([0,+\infty);D(\rA))\cap C^1([0,+\infty);X)\]

Se {$\xo\not\in D(A)$} em $X$, então $x(t)=S(t)\xo$ {\color{red}não é diferenciável}. Neste caso, dizemos que $x=x(t)$ é uma {\color{blue}solução generalizada (fraca)} do PVI.
\end{exampleblock}

\end{frame}

\begin{frame}{ }
\begin{teo}
Seja $S(t)$ um semigrupo de classe $C^0$ em $X$ e $\rA$ o seu gerador infinitesimal. Dado $x\in D(\rA)$, então 
\[S{(\cdot)\rA}x\in C^0([0,+\infty);D(\rA))\cap C^1([0,+\infty);X)\]
e
\[\frac{d}{dt}\left(S(t)x\right)=\rA S(t)x=S(t)\rA x.\]
\end{teo}

\begin{exer}
Seja $S(t)$ um semigrupo de classe $C^0$ em $X$ e $\rA$ o seu gerador infinitesimal. Se $x\in D(\rA)$ mostre que
\[S(t)x-S(s)x=\int_s^t AS(\xi)x\,d\xi =\int_s^t S(\xi)Ax\,d\xi\]
\end{exer}
\end{frame}

\begin{frame}{ }
\begin{prop}
Se $S(t)=e^{tA}$ em $X$, então para todo $x\in X$, 
\[\int_0^t e^{sA}x\,ds\in D(A)\ \text{ e }\ A\left(\int_0^t e^{sA}x\,ds\right)=e^{tA}x-x.\]
\end{prop}

\begin{prop}
Se $S(t)=e^{tA}$ em $X$, então $A$ é fechado e seu domínio é denso em $X$.
\end{prop}

\begin{exampleblock}{}
Esta última proposição nos dá uma condição necessária para que um operador $A$ seja o gerador infinitesimal de um semigrupo.
\end{exampleblock}
\begin{prop}[Unicidade]
Se $S_1(t)=e^{tA}$ e $S_2(t)=e^{tA}$ em $X$, então $S_1=S_2$.
\end{prop}
\end{frame}


